%% !TEX encoding = UTF-8 Unicode

\documentclass[12pt]{article}

\usepackage[a4paper, left=2cm, right=2cm, top=2cm]{geometry}
\usepackage[utf8]{inputenc}
\usepackage[onehalfspacing]{setspace} %1.5px Zeilenabstand
\usepackage[ngerman]{babel}
\usepackage{graphicx} % Grafikpaket laden
\usepackage{mathptmx} %Times New Roman
\usepackage{apacite} %APA 6
\usepackage[hyphens]{url}
\usepackage{acronym} %Abkürzungen
\usepackage{tocstyle}  % Für Punkte im Inhaltsverzeichnis
\usepackage{multicol}

\newcommand\tab[1][1cm]{\hspace*{#1}}
\newcommand*\wildcard[2][5cm]{\vspace*{2cm}\parbox{#1}{\hrulefill\par#2}}
\newcommand*{\quelle}{%
  \footnotesize Quelle: 
} 

\begin{document}

% Deckblatt
% Titel Seite
	\begin{titlepage}

		\begin{center}
		\line(1,0){300} \\
		[2mm]
		\huge{\bfseries Abschlussbericht} \\
		\line(1,0){300} \\
		[1cm]

		\large{im Studiengang \\ Informationswissenschaft \& Sprachtechnologie} \\
		[1cm]
		\LARGE{\textbf{DINO Projekt - Coding Team}} \\
		[1.5cm]

		\large{\textbf {vorgelegt von}}\\ 

	\end{center}
	
	
\end{titlepage}
% Ende Titel Seite

% Inhaltsverzeichnis
\newpage
\newtocstyle[KOMAlike][leaders]{alldotted}{} %Gepunktete Linien
\usetocstyle{alldotted} % Gepunktete Linien
\tableofcontents
% Ende Inhaltsverzeichnis

% Abkürzungsverzeichnis
\newpage
\section*{Abkürzungsverzeichnis}
\addcontentsline{toc}{section}{Abkürzungsverzeichnis} %Section ohne Nummer im Inhaltsverzeichnis
\begin{acronym}[WLAN]	%USA als längste Abkürzung Def. für Tab Abstand
	\setlength{\itemsep}{-\parsep} % kein Abstand, kompakte Darstellung 
	\acro{dino}[DINO]{Düsseldorf Information Nachhaltigkeit Offenheit}

\end{acronym}

% Abbildungsverzeichnis
\listoffigures
\newpage

% Literaturverzeichnis
\newpage
\bibliographystyle{apacite}
\bibliography{Literaturverzeichnis}
	

\end{document}
